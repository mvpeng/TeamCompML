% Template file for an a0 landscape poster.
% Written by Graeme, 2001-03 based on Norman's original microlensing
% poster.
%
% See discussion and documentation at
% <http://www.astro.gla.ac.uk/users/norman/docs/posters/> 
%
% $Id: poster-template-landscape.tex,v 1.2 2002/12/03 11:25:46 norman Exp $


% Default mode is landscape, which is what we want, however dvips and
% a0poster do not quite do the right thing, so we end up with text in
% landscape style (wide and short) down a portrait page (narrow and
% long). Printing this onto the a0 printer chops the right hand edge.
% However, 'psnup' can save the day, reorienting the text so that the
% poster prints lengthways down an a0 portrait bounding box.
%
% 'psnup -w85cm -h119cm -f poster_from_dvips.ps poster_in_landscape.ps'

\documentclass[a0]{a0poster}
% You might find the 'draft' option to a0 poster useful if you have
% lots of graphics, because they can take some time to process and
% display. (\documentclass[a0,draft]{a0poster})
\input defs
\pagestyle{empty}
\setcounter{secnumdepth}{0}
\renewcommand{\familydefault}{\sfdefault}
\newcommand{\QED}{~~\rule[-1pt]{8pt}{8pt}}\def\qed{\QED}

\renewcommand{\reals}{{\mbox{\bf R}}}

% The textpos package is necessary to position textblocks at arbitary 
% places on the page.
\usepackage[absolute]{textpos}

\usepackage{fleqn,psfrag,wrapfig,tikz,amsmath, framed, scrextend,subcaption}
\usepackage{mathtools,url}
\DeclarePairedDelimiterX{\norm}[1]{\lVert}{\rVert}{#1}

\usepackage[papersize={38in,28in}]{geometry}

% Graphics to include graphics. Times is nice on posters, but you
% might want to switch it off and go for CMR fonts.
\usepackage{graphics}

\renewenvironment{leftbar}[1][\hsize]
{% 
\def\FrameCommand 
{%

    {\color{black}\vrule width 0pt}%
    \hspace{0pt}%must no space.
    \fboxsep=\FrameSep\colorbox{white}%
}%
\MakeFramed{\hsize#1\advance\hsize-\width\FrameRestore}%
}
{\endMakeFramed}

% we are running pdflatex, so convert .eps files to .pdf
%\usepackage[pdftex]{graphicx}
%\usepackage{epstopdf}

% These colours are tried and tested for titles and headers. Don't
% over use color!
\usepackage{color}
\definecolor{Red}{rgb}{0.9,0.0,0.1}

\definecolor{bluegray}{rgb}{0.15,0.20,0.40}
\definecolor{bluegraylight}{rgb}{0.35,0.40,0.60}
\definecolor{gray}{rgb}{0.3,0.3,0.3}
\definecolor{lightgray}{rgb}{0.7,0.7,0.7}
\definecolor{darkblue}{rgb}{0.2,0.2,1.0}
\definecolor{darkgreen}{rgb}{0.0,0.5,0.3}

\renewcommand{\labelitemi}{\textcolor{bluegray}\textbullet}
\renewcommand{\labelitemii}{\textcolor{bluegray}{--}}

\setlength{\labelsep}{0.5em}


% see documentation for a0poster class for the size options here
\let\Textsize\normalsize
%\def\Head#1{\noindent\hbox to \hsize{\hfil{\LARGE\color{bluegray} #1}}\bigskip}
\def\Head#1{\noindent{\LARGE\color{bluegray} #1}\bigskip}
\def\LHead#1{\noindent{\LARGE\color{bluegray} #1}\bigskip}
\def\Subhead#1{\noindent{\large\color{bluegray} #1}\bigskip}
\def\Title#1{\noindent{\VeryHuge\color{Red} #1}}

\usepackage{multicol}
\setlength{\columnsep}{1cm}
\usepackage{vwcol} 
\usepackage{booktabs}

% Set up the grid
%
% Note that [40mm,40mm] is the margin round the edge of the page --
% it is _not_ the grid size. That is always defined as 
% PAGE_WIDTH/HGRID and PAGE_HEIGHT/VGRID. In this case we use
% 23 x 12. This gives us three columns of width 7 boxes, with a gap of
% width 1 in between them. 12 vertical boxes is a good number to work
% with.
%
% Note however that texblocks can be positioned fractionally as well,
% so really any convenient grid size can be used.
%
\TPGrid[40mm,40mm]{23}{12}      % 3 cols of width 7, plus 2 gaps width 1

\parindent=0pt
\parskip=0.2\baselineskip

\begin{document}

% Understanding textblocks is the key to being able to do a poster in
% LaTeX. In
%
%    \begin{textblock}{wid}(x,y)
%    ...
%    \end{textblock}
%
% the first argument gives the block width in units of the grid
% cells specified above in \TPGrid; the second gives the (x,y)
% position on the grid, with the y axis pointing down.

% You will have to do a lot of previewing to get everything in the 
% right place.

% This gives good title positioning for a portrait poster.
% Watch out for hyphenation in titles - LaTeX will do it
% but it looks awful.
\begin{textblock}{23}(0,0)
\Title{Player Behavior and Optimal Team Compositions in Online Games}
\end{textblock}

\begin{textblock}{23}(0,0.6)
{
\LARGE
Hao Yi Ong,
Sunil Deolalikar, and
Mark Peng
}

{
\Large
\color{bluegray}
\emph{CS 229: Machine Learning Class Project}
}
\end{textblock}


% Uni logo in the top right corner. A&A in the bottom left. Gives a
% good visual balance, but you may want to change this depending upon
% the graphics that are in your poster.
%\begin{textblock}{2}(0,10)
%Your logo here
%%\includegraphics{/usr/local/share/images/AandA.epsf}
%\end{textblock}

% \begin{textblock}{3}(21.0,0)
% % Another logo here
% \resizebox{1.95\TPHorizModule}{!}{\includegraphics{uni-logo.png}}
% \end{textblock}


\begin{textblock}{7.0}(0,1.5)

\hrule\medskip
\Head{Introduction}

\begin{itemize}
  
  \item In online role-playing games, players work in teams to accomplish a common objective (e.g., defeating an opposing team)

\end{itemize}

\begin{figure}[!h]
  \centering
  \includegraphics[width=\textwidth]{intro-fig.pdf}
  \label{fig:intro}
\end{figure}

\vspace{-1em}

\begin{itemize}

  \item In a game, given the teams' player compositions and their player statistics, we want to predict the players' play style and forecast the game outcome

\end{itemize}

\begin{multicols}{2}
  
  \begin{itemize}
    \item \textbf{Player behavior}
    \begin{itemize}
      \item In-game play style; e.g., prefers more offense-oriented strategies
      \item Also encompasses skill level
      \item Predict from player statistics
    \end{itemize}

    \item \textbf{Team composition}
    \begin{itemize}
      \item Types of players on a team, each classified by their play styles
      \item Predict from game server database's match histories
    \end{itemize}
  \end{itemize}

\end{multicols}

\medskip
\hrule\medskip
\Head{Problem Description}

\begin{itemize}

  \item Given
  \begin{itemize}
    \item \textbf{Match histories} containing participant IDs and match statistics
    \item \textbf{Player statistics} containing player histories and overall game statistics
  \end{itemize}

  \item Output
  \begin{itemize}
    \item \textbf{Play style classifier} that groups players by their in-game tendencies given their game histories
    \item \textbf{Outcome predictor} that guesses which team will win given the various team compositions
  \end{itemize}

  \item In order to
  \begin{itemize}
    \item \textbf{Gain insight} on player behaviors and game strategies
    \item \textbf{Maximize accuracy} on predicting game outcomes
  \end{itemize} 

\end{itemize}

\medskip
\hrule\medskip
\Head{Numerical Simulation}

% \begin{multicols}{1}
  \begin{itemize}
    
    \item \textbf{Target Game: League of Legends}
    \begin{itemize}
      \item Multiplayer battle arena game with 27 million plays per day
      \item Free online API to retrieve recorded game data
      \item Official guide provides clustering information for players based on in-game character choices (e.g., character with good defense)
    \end{itemize}

    % \begin{figure}[!h]
    %   \centering
    %   \includegraphics[width=0.25\textwidth]{lol-icon.pdf}
    %   \label{fig:lol}
    % \end{figure}

    \item \textbf{Data samples} Total of 120,000 training and 12,000 test samples

    \item \textbf{Implementation} 
    \begin{itemize} 
      \item Clustering and classification algorithms in MATLAB 2014b
      \item Data processing and feature selection in Python 2.7
    \end{itemize}

    \item \textbf{Hardware} All simulations on 2.7 GHz Intel Core i7, 8 GB RAM

  \end{itemize}
% \end{multicols}

\end{textblock}

\begin{textblock}{7.0}(8,1.5)

\hrule\medskip
\Head{Baseline Outcome Predictor}

\begin{itemize}
  
  \item \textbf{Features} are team compositions for all teams in a game based on official guide's clustering information (characters are mapped to 1 of 5 play styles)

  \item \textbf{Logistic regression} with 10\% hold-out cross validation
  
  \item \textbf{Poor accuracy} of 55.1\% on training samples, 54.4\% on test samples

\end{itemize}

\medskip
\hrule\medskip
\Head{Behavioral Clustering}

\begin{itemize}
  
  \item \textbf{Features} are normalized player statistics (damage dealt, money earned,...)

  \item Clustering algorithms (unsupervised learning)
  \begin{itemize}
    
    \item \textbf{k-means} with 10-fold cross validation over parameter k gave 12 clusters
    
    \item \textbf{DP-means} is a nonparametric expectation-maximization algorithm derived using a Dirichlet process mixture model (Kulis and Jordan, 2012)

    \item Intuitively, a new cluster is formed whenever a point is sufficiently far away from all existing centroids, as determined by some threshold distance \(\lambda\)

    \item We ran it with 10\% hold-out cross validation with \(\lambda\) = 3.3, giving 8 clusters

    \begin{addmargin}[0em]{2em}% 1em left, 2em right
      \begin{leftbar}
        \begin{tabbing}

          {\bf given} training set of size \(N\), threshold distance \(\lambda\) \\*[\smallskipamount]
          {\bf repeat} \\
          \qquad \= For \(n\) = 1, \(\ldots\) , \(N\) \\
          \qquad \qquad \= 1.\ Assign sample \(n\) to the closest cluster if the contribution to \\ \hspace{113pt} objective  from the squared distance is at most \(\lambda^{2}\) \\
          \> 2.\ Otherwise, form a new cluster with just sample \(n\) \\
          {\bf until} clusters converge
        \end{tabbing}
      \end{leftbar}
    \end{addmargin}

  \end{itemize}

  \medskip

  \begin{table}[htbp!]
    \begin{minipage}{\textwidth}
      \centering
      \caption*{\textbf{Summary:} Play style clustering algorithm results (10 trials)}
      \begin{tabular}{lcccc}
        \toprule
        & cross validation method & no. of clusters & cpu time (s) \\ \midrule
        k-means & k-fold (k = 10) & 12 & 154.1 \\
        DP-means (\(\lambda\) = 3.3) & 10\% hold-out & 8 & 65.4 \\
        \bottomrule
      \end{tabular}
      \label{table:cluster}
    \end{minipage} 
  \end{table}

\end{itemize}

\medskip
\hrule\medskip
\Head{Cluster Intepretation}

\begin{itemize}

  \item Surprisingly accurate intuition behind groups (with input from expert players)

  \begin{multicols}{2}

    \begin{itemize}
      
      \item \textbf{Physical damage attacker}
      \begin{itemize}
        \item Clusters 1, 7, and 9
        \item Differ in risk attitudes
      \end{itemize}
      
      \item \textbf{Ambusher}
      \begin{itemize}
        \item Clusters 3, 8, 11, and 12
        \item Team oriented vs. lone wolf
        \item Includes ``hybrid'' roles with other play style clusters
      \end{itemize}
      
      \item \textbf{Team support}
      \begin{itemize}
        \item Cluster 5: Many assists in kills
      \end{itemize}
      
      \item \textbf{Magic attacker}
      \begin{itemize}
        \item Clusters 6 and 10: Many kills
        \item Ranged- vs. close-combat
      \end{itemize}
      
      \item \textbf{Miscellaneous}
      \begin{itemize}
        \item Cluster 2: All-around average
        \item Cluster 4: Novice player
      \end{itemize}

    \end{itemize}

  \end{multicols}

\end{itemize}

\begin{figure}[!h]
  \centering
  \includegraphics[width=0.9\textwidth]{clusters-intuit.pdf}
  \label{fig:intuit}
\end{figure}

\end{textblock}


\begin{textblock}{7.0}(16,1.5)

\hrule\medskip
\Head{Cluster Visualization}

\begin{itemize}
  \item \textbf{Principal component analysis} helps us visualize our dataset in 3D

  \begin{figure}[!h]
    \centering
    \includegraphics[trim=60pt 210pt 70pt 250pt, clip, width=0.75\textwidth]{viz-cluster.pdf}
    \label{fig:viz}
  \end{figure}
\end{itemize}

\medskip
\hrule\medskip
\Head{Game Outcome Prediction}

\begin{itemize}

  \item \textbf{Features} are team compositions for all teams in a game based on behavioral groupings generated from k-means and DP-means clustering

  \item \textbf{Labels} are win/loss for each game (if team 1 \emph{beats} team 2, label = 1)

  \item Classification algorithms (supervised learning)
  \begin{itemize}
    
    \item \textbf{Logistic regression} generalized linear model with Bernoulli distribution

    \item \textbf{Gaussian discriminant analysis} assuming data is Gaussian-distributed

    \item \textbf{Support vector machine} assuming data is separable by soft-margins

    \item \textbf{Cross validation} 10\% hold-out for all methods

  \end{itemize}

  \item \textbf{Best predictor} uses features based on k-means clustered team compositions for each game, trained on an SVM; 70.4\% accuracy (vs. 55.1\% baseline)

\end{itemize}

\medskip

\begin{table}[htbp!]
  \begin{minipage}{\textwidth}
    \centering
    \caption*{\textbf{Summary:} Outcome prediction algorithm results (10 trials)}
    \begin{tabular}{lcccccc}
      \toprule
      & \multicolumn{2}{c}{k-means (\%)} & \multicolumn{2}{c}{DP-means (\%)} & \multicolumn{2}{c}{cpu time (s)} \\
      & train acc. & test acc. & train acc. & test acc. & k-means & DP-means \\ \midrule
      LR & 72.25 & 68.75 & 69.67 & 67.11 & 7.4 & 7.1 \\
      GDA & 74.79 & 70.14 & 70.88 & 68.39 & 7.7 & 7.1 \\
      SVM & 74.75 & 70.39 & 71.71 & 69.21 & 91.2 & 41.6 \\
      \bottomrule
    \end{tabular}
    \label{table:summary}
  \end{minipage} 
\end{table}

\medskip
\hrule\medskip
\Head{Conclusion and Extensions}

\begin{itemize}
  
  \item k-means and DP-means provided clusters separated by highly intuitive play style differences, as confirmed by expert players

  \item LR, GDA, and SVM all provided better predictors with our new team composition features than baseline LR with official character classes

  \item More features, such as time-identified statistics (early vs. late game actions)

\end{itemize}

\medskip
\hrule\medskip
\Head{Acknowledgments}

We thank Professor Andrew Ng and the instructor team.

\end{textblock}

\end{document}

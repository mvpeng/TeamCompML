% Template file for an a0 landscape poster.
% Written by Graeme, 2001-03 based on Norman's original microlensing
% poster.
%
% See discussion and documentation at
% <http://www.astro.gla.ac.uk/users/norman/docs/posters/> 
%
% $Id: poster-template-landscape.tex,v 1.2 2002/12/03 11:25:46 norman Exp $


% Default mode is landscape, which is what we want, however dvips and
% a0poster do not quite do the right thing, so we end up with text in
% landscape style (wide and short) down a portrait page (narrow and
% long). Printing this onto the a0 printer chops the right hand edge.
% However, 'psnup' can save the day, reorienting the text so that the
% poster prints lengthways down an a0 portrait bounding box.
%
% 'psnup -w85cm -h119cm -f poster_from_dvips.ps poster_in_landscape.ps'

\documentclass[a0]{a0poster}
% You might find the 'draft' option to a0 poster useful if you have
% lots of graphics, because they can take some time to process and
% display. (\documentclass[a0,draft]{a0poster})
\input defs
\pagestyle{empty}
\setcounter{secnumdepth}{0}
\renewcommand{\familydefault}{\sfdefault}
\newcommand{\QED}{~~\rule[-1pt]{8pt}{8pt}}\def\qed{\QED}

\renewcommand{\reals}{{\mbox{\bf R}}}

% The textpos package is necessary to position textblocks at arbitary 
% places on the page.
\usepackage[absolute]{textpos}

\usepackage{fleqn,psfrag,wrapfig,tikz,amsmath, framed, scrextend,subcaption}
\usepackage{mathtools,url}
\DeclarePairedDelimiterX{\norm}[1]{\lVert}{\rVert}{#1}

\usepackage[papersize={38in,28in}]{geometry}

% Graphics to include graphics. Times is nice on posters, but you
% might want to switch it off and go for CMR fonts.
\usepackage{graphics}

\renewenvironment{leftbar}[1][\hsize]
{% 
\def\FrameCommand 
{%

    {\color{black}\vrule width 0pt}%
    \hspace{0pt}%must no space.
    \fboxsep=\FrameSep\colorbox{white}%
}%
\MakeFramed{\hsize#1\advance\hsize-\width\FrameRestore}%
}
{\endMakeFramed}

% we are running pdflatex, so convert .eps files to .pdf
%\usepackage[pdftex]{graphicx}
%\usepackage{epstopdf}

% These colours are tried and tested for titles and headers. Don't
% over use color!
\usepackage{color}
\definecolor{Red}{rgb}{0.9,0.0,0.1}

\definecolor{bluegray}{rgb}{0.15,0.20,0.40}
\definecolor{bluegraylight}{rgb}{0.35,0.40,0.60}
\definecolor{gray}{rgb}{0.3,0.3,0.3}
\definecolor{lightgray}{rgb}{0.7,0.7,0.7}
\definecolor{darkblue}{rgb}{0.2,0.2,1.0}
\definecolor{darkgreen}{rgb}{0.0,0.5,0.3}

\renewcommand{\labelitemi}{\textcolor{bluegray}\textbullet}
\renewcommand{\labelitemii}{\textcolor{bluegray}{--}}

\setlength{\labelsep}{0.5em}


% see documentation for a0poster class for the size options here
\let\Textsize\normalsize
%\def\Head#1{\noindent\hbox to \hsize{\hfil{\LARGE\color{bluegray} #1}}\bigskip}
\def\Head#1{\noindent{\LARGE\color{bluegray} #1}\bigskip}
\def\LHead#1{\noindent{\LARGE\color{bluegray} #1}\bigskip}
\def\Subhead#1{\noindent{\large\color{bluegray} #1}\bigskip}
\def\Title#1{\noindent{\VeryHuge\color{Red} #1}}

\usepackage{multicol}
\setlength{\columnsep}{1cm}
\usepackage{vwcol} 

% Set up the grid
%
% Note that [40mm,40mm] is the margin round the edge of the page --
% it is _not_ the grid size. That is always defined as 
% PAGE_WIDTH/HGRID and PAGE_HEIGHT/VGRID. In this case we use
% 23 x 12. This gives us three columns of width 7 boxes, with a gap of
% width 1 in between them. 12 vertical boxes is a good number to work
% with.
%
% Note however that texblocks can be positioned fractionally as well,
% so really any convenient grid size can be used.
%
\TPGrid[40mm,40mm]{23}{12}      % 3 cols of width 7, plus 2 gaps width 1

\parindent=0pt
\parskip=0.2\baselineskip

\begin{document}

% Understanding textblocks is the key to being able to do a poster in
% LaTeX. In
%
%    \begin{textblock}{wid}(x,y)
%    ...
%    \end{textblock}
%
% the first argument gives the block width in units of the grid
% cells specified above in \TPGrid; the second gives the (x,y)
% position on the grid, with the y axis pointing down.

% You will have to do a lot of previewing to get everything in the 
% right place.

% This gives good title positioning for a portrait poster.
% Watch out for hyphenation in titles - LaTeX will do it
% but it looks awful.
\begin{textblock}{23}(0,0)
\Title{Player Behavior and Optimal Team Compositions in Online Games}
\end{textblock}

\begin{textblock}{23}(0,0.6)
{
\LARGE
Hao Yi Ong,
Sunil Deolalikar, and
Mark Peng
}

{
\Large
\color{bluegray}
\emph{CS 229: Machine Learning Class Project}
}
\end{textblock}


% Uni logo in the top right corner. A&A in the bottom left. Gives a
% good visual balance, but you may want to change this depending upon
% the graphics that are in your poster.
%\begin{textblock}{2}(0,10)
%Your logo here
%%\includegraphics{/usr/local/share/images/AandA.epsf}
%\end{textblock}

% \begin{textblock}{3}(21.0,0)
% % Another logo here
% \resizebox{1.95\TPHorizModule}{!}{\includegraphics{uni-logo.png}}
% \end{textblock}


\begin{textblock}{7.0}(0,1.5)

\hrule\medskip
\Head{Introduction}

\begin{itemize}
  
  \item In online role-playing games, players work in teams to accomplish a common objective (e.g., defeating an opposing team)

\end{itemize}

\begin{figure}[!h]
  \centering
  \includegraphics[width=\textwidth]{intro-fig.pdf}
  \label{fig:intro}
\end{figure}

\begin{itemize}

  \item In a game, given the teams' player compositions and their player statistics, we want to predict the players' play style and forecast the game outcome

\end{itemize}

\begin{multicols}{2}
  
  \begin{itemize}
    \item \textbf{Player behavior}
    \begin{itemize}
      \item In-game play style; e.g., prefers more offense-oriented strategies
      \item Also encompasses skill level
      \item Predict from player statistics
    \end{itemize}

    \item \textbf{Team composition}
    \begin{itemize}
      \item Types of players on a team, each classified by their play styles
      \item Predict from game server database's match histories
    \end{itemize}
  \end{itemize}

\end{multicols}

\medskip
\hrule\medskip
\Head{Problem Description}

\begin{itemize}

  \item Given
  \begin{itemize}
    \item \textbf{Match histories} containing participant IDs and match statistics
    \item \textbf{Player statistics} containing player histories and overall game statistics
  \end{itemize}

  \item Output
  \begin{itemize}
    \item \textbf{Play style classifier} that groups players by their in-game tendencies given their game histories
    \item \textbf{Outcome predictor} that guesses which team will win given the various team compositions
  \end{itemize}

  \item In order to
  \begin{itemize}
    \item \emph{Gain insight} on player behaviors and game strategies
    \item \emph{Maximize accuracy} on predicting game outcomes
  \end{itemize} 

\end{itemize}

\medskip
\hrule\medskip
\Head{Target Game}

\begin{multicols}{2}
  \begin{itemize}
    
    \item \textbf{League of Legends}
    \begin{itemize}
      \item Multiplayer battle arena game with 27 million plays per day
      \item Free online API to retrieve de-identified game data
      \item Official guide provides clustering information for players based on in-game character choices (e.g., character with good defense)
    \end{itemize}

    \vspace{30pt}

    \begin{figure}[!h]
      \centering
      \includegraphics[width=0.30\textwidth]{lol-icon.pdf}
      \label{fig:lol}
    \end{figure}

  \end{itemize}
\end{multicols}

\end{textblock}

\begin{textblock}{7.0}(8,1.5)

\hrule\medskip
\Head{Baseline Outcome Predictor}

\begin{itemize}
  
  \item \textbf{Features} are team compositions for each team in a game based on official game guide's clustering information (each character is mapped to a play style)

  \item \textbf{Logistic regression} with 10\% hold-out cross validation
  
  \item \textbf{Data samples} with 120,000 training samples and 12,000 test samples
  
  \item \textbf{Poor accuracy} of 55.1\% on training samples, 54.4\% on test samples

\end{itemize}

\medskip
\hrule\medskip
\Head{Behavioral Clustering}



\medskip
\hrule\medskip
\Head{Cluster Visualization}



\end{textblock}

  
\begin{textblock}{7.0}(16,1.5)

\hrule\medskip
\Head{Results and Analysis}

\begin{itemize}
  
  \item \textbf{Best structure} We picked the top 8 structures based on the Bayesian score, ran parameter learning on them, and tested their prediction accuracies based on classifying test sets using inference

\end{itemize}

\end{textblock}

\begin{textblock}{7.0}(16,2.75)

\begin{figure}[!h]
  \centering
  \hbox{
    \hspace{-2em}
    % \includegraphics[width=1.1\textwidth]{bn-struct.pdf}
  }
  \label{fig3}
\end{figure}

\end{textblock}

\begin{textblock}{7.0}(16,5.25)

\begin{itemize}

  \item \textbf{Error rates} Varying the number of missing variables, the error rates were 15--19\% on 1,000 test samples, which are competitive with the 16\% baseline

  \item \textbf{Runtime} Depending on the number of missing variables, a single prediction can take up to 30 seconds---much more expensive than our baseline

\end{itemize}

\end{textblock}

\begin{textblock}{7.0}(16,6.2)

\begin{figure}[!h]
  \centering
  % \includegraphics[trim=3pt 3pt 3pt 10pt, clip, width=0.8\textwidth]{../src/Excel/BayesStructureChart.pdf}
  \label{fig3}
\end{figure}

\begin{table}[htbp!]
  \begin{minipage}{\textwidth}
    \centering
    \caption*{Summary: Classification results with varying number of missing features}
    \begin{tabular}{|c||c|c|c|c|}
      \hline
      \#missing & Error rate & False positive & False negative & Pred time (sec) \\ \hline \hline
      3 & 0.1662 & 0.0363 & 0.1299 & 3.754 \\ \hline
      5 & 0.1511 & 0.0302 & 0.1208 & 6.437 \\ \hline
      8 & 0.1339 & 0.0272 & 0.1067 & 9.672 \\ \hline
      10 & 0.1299 & 0.0252 & 0.1047 & 11.859 \\ \hline
    \end{tabular}
    \label{table:summary}
  \end{minipage} 
\end{table}

\medskip
\hrule\medskip
\Head{Conclusion and Extensions}

\begin{itemize}
  
  \item Our model is suitable for diabetes prediction on existing databases (even with incomplete records) to detect patients at risk for diabetes
  
  \item Incorporate expert knowledge in feature selection and deciding the structure
  
  \item Consider different inference methods for runtime and accuracy improvement

\end{itemize}

\medskip
\hrule\medskip
\Head{Acknowledgements}

We thank Professor Andrew Ng, the instructor team, and fellow students for their help with and feedback on our work.

\end{textblock}

\end{document}
